\documentclass[24]{article}

\usepackage{setspace}

\usepackage{fontspec}
%\usepackage{arabxetex}
\usepackage{polyglossia}

\setmainlanguage{english}
\setotherlanguage{arabic}

\newfontfamily\arabicfont[Script=Arabic,Scale=4.5]{UKIJ Tuz}%{Scheherazade}

\begin{document}
\textarabic{كومپيوتېر كۇۋادىرات يىلتىزنى قانداق ھىساپلايدۇ؟ 123}

\newfontfamily\arabicfont[Script=Arabic,Scale=1.5]{UKIJ Merdane}%{Scheherazade}
\begin{Arabic}

كومپيوتېردا كۇۋادىرات يىلتىزنى ھىساپلاشنىڭ كۆپلىگەن ئۇسۇللىرى بار. نىيوتون ئۇسۇلى شۇلارنىڭ ئىچىدىكى بىرى. 
\end{Arabic}

In numerical analysis, Newton's method (also known as the Newton–Raphson method), named after Isaac Newton and Joseph Raphson, is a method for finding successively better approximations to the roots (or zeroes) of a real-valued function.

    x : f(x) = 0,.

The algorithm is first in the class of Householder's methods, succeeded by Halley's method. The method can also be extended to complex functions and to systems of equations.

The Newton-Raphson method in one variable is implemented as follows:

Given a function ƒ defined over the reals x, and its derivative ƒ ', we begin with a first guess x0 for a root of the function f. Provided the function satisfies all the assumptions made in the derivation of the formula, a better approximation x1 is
\begin{equation}
    x_{1} = x_0 - \frac{f(x_0)}{f'(x_0)}.
\end{equation}
Geometrically, (x1, 0) is the intersection with the x-axis of a line tangent to f at (x0, f (x0)).

The process is repeated as
\begin{equation}
    x_{n+1} = x_n - \frac{f(x_n)}{f'(x_n)} ,
\end{equation}

until a sufficiently accurate value is reached.


Examples
Square root of a number

Consider the problem of finding the square root of a number. There are many methods of computing square roots, and Newton's method is one.

For example, if one wishes to find the square root of 612, this is equivalent to finding the solution to

$    x^2 = 612$

The function to use in Newton's method is then,

   $ f(x) = x^2 - 612$

with derivative,

 $   f'(x) = 2x. , $

With an initial guess of 10, the sequence given by Newton's method is

 % \begin{matrix} x_1 \& = \& x_0 - \dfrac{f(x_0)}{f'(x_0)} & = & 10 - \dfrac{10^2 - 612}{2 \cdot 10} & = & 35.6 \quad\quad\quad{} \\ x_2 & = & x_1 - \dfrac{f(x_1)}{f'(x_1)} & = & 35.6 - \dfrac{35.6^2 - 612}{2 \cdot 35.6} & = & \underline{2}6.395505617978\dots \\ x_3 & = & \vdots & = & \vdots & = & \underline{24.7}90635492455\dots \\ x_4 & = & \vdots & = & \vdots & = & \underline{24.7386}88294075\dots \\ x_5 & = & \vdots & = & \vdots & = & \underline{24.7386337537}67\dots \end{matrix} 

Where the correct digits are underlined. With only a few iterations one can obtain a solution accurate to many decimal places.



%\end{Arabic}
\end{document}